% JuliaCon proceedings template
\documentclass{juliacon}
\usepackage{amsmath}
\setcounter{page}{1}

\begin{document}
% **************GENERATED FILE, DO NOT EDIT**************

\title{My JuliaCon proceeding}

\author[1]{1st author}
\author[1, 2]{2nd author}
\author[2]{3rd author}
\affil[1]{University}
\affil[2]{National Lab}

\keywords{Julia, Optimization, Game theory, Compiler}


\maketitle
\begin{abstract}

The aim of this paper is twofold. The first one is to describe a novel research-project designed for building bridges between machine learning and econometric worlds. The second one is to introduce the main characteristics and comparative performance of the first Julia-native all-subset regression algorithm included in \verb GlobalSearchRegression.jl  (v1.0.3). As other available alternatives, this algorithm allows researchers to obtain the best model specification among all possible covariate combinations - in terms of user defined information criteria-, but up to 3165 and 197 times faster than STATA and R alternatives, respectively. 

\end{abstract}

\section{Introduction}
While allowing for better and more accurate analysis (and forecasts), fat-data availability generates a revival of Bellman’s \cite{Bellman1961} -and newer- high-dimensionality concerns:
\begin{enumerate}
    \item Exponential increase in required information and execution time \cite{Yu2003} ;
    \item Overfitting risk \cite{Defernez1999}; 
    \item Less-informative euclidean distances \cite{Aggarwal2001}; and
    \item Unfeasible estimators \cite{Buhlmann2011}. 
\end{enumerate}

Notwithstanding, the advantage of having thousands/millions of features to deal with complex phenomena stimulates an unprecedented number of methodological -and technological- improvements to manage the ‘curse of dimensionality’ \cite{Bolon2016}. \vskip 6pt

In Economics, this process has a dual approach with machine-learning (ML) and econometric (EC) algorithms emerging for different purposes: the former for prediction/forecasts (focusing on $\hat{y}$) and the latter for estimation/causal inference (interested in $\hat{\beta}$). Alternatively, the same distinction can be expressed in Diebold’s terms as non-causal vs causal prediction (see \cite{athey2015}), where ML algorithms are designed to reduce prediction sampling-risks -i.e. learning through cross-validation techniques- and EC methods to identify unbiased multivariate relationships -i.e. avoiding consistency issues through residual and coefficient tests for model selection. \vskip 6pt

In turn, ML feature selection algorithms can be classified into three different families: Filters, Wrappers and Embedded -depending on whether they use some classifier/response variable information or not, or how variable selection is made along with the learning process, see \cite{chandras2014}. Similarly, most EC dimensionality reduction approaches can be classified into three different groups: Exhaustive, General-to-Specific and Specific-to-General -depending on the search pattern; see \cite{gluzmann2015}. \vskip 6pt

Despite significant improvements in recent econometric developments -like PCGIVE/Autometrics or Retina algorithms, which combine some ML and EC characteristics-, available alternatives fails to fully exploit cross-validation and model averaging capabilities -see \cite{Doornik09autometrics}, \cite{perezamaral2003}, \cite{DAVIDSON1981}, \cite{derksen1992}, \cite{marinucci2008}, \cite{herwartz2010}, and \cite{castle2006}. \vskip 6pt

Conversely, newer ML algorithms -like Convolutional Neural Networks or Bootstrap-Based LASSO, which improve complex non-linear adjustment or model selection under regularization schemes- achieved unprecedented forecast accuracy but disregarding model interpretability and/or parameter estimation issues -i.e omitting residual and coefficient tests; see \cite{Bzdok2018}. \vskip 6pt

Following Varian’s \cite{varian2014} advices, about ML and EC complementarities -i.e. merging algorithms from different families to reduce both sampling and model uncertainty-, we are developing a novel multi-layer-multi-algorithm methodology combining two reinforcing paradigms: The LSE “Testimation” approach -to obtain information about residual properties, see \cite{abramovich2006}- and the Bayesian-like “Double-model averaging” -across different covariates and sub-samples, see \cite{hoetening1999}. This methodology includes five complementary layers -handling cross-section, time series and panel data-: 1) Pre-processing: with outlier detection, missing values identification, seasonal adjustment and normalization/standardization functions; 2) Feature extraction: creation of logs, squares, inverses and interactions from selected variables; 3) Feature pre-selection: using filter and embedded ML algorithms like CFS, Variance threshold and LASSO functions; 4) Final feature selection: with a modified all-subset regression approach, including residual tests and model averaging capabilities; 5) Post-estimation fine-tuning: coefficient re-evaluation through cross-validation techniques and model averaging across different k-fold results. \vskip 6pt


The objective of this paper is to introduce the main characteristics and comparative performance of a key layer of our methodology: the modified all-subset 
regression algorithm included in \href{https://github.com/ParallelGSReg/GlobalSearchRegression.jl}{GlobalSearchRegression.jl} (v1.0.3). As other available 
alternatives (like \href{https://cran.r-project.org/web/packages/MuMIn/MuMIn.pdf}{MuMin-pdredge} in R, or \href{https://www.researchgate.net/profile/
Pablo_Gluzmann/publication/264782750_Global_Search_Regression_A_New_Automatic_Model-selection_Technique_for_Cross-section_Time-series_and_Panel-data_Regressions/
links/53eed18a0cf23733e812c10d/Global-Search-Regression-A-New-Automatic-Model-selection-Technique-for-Cross-section-Time-series-and-Panel-data-Regressions.pdf?
origin=publication_detail}{GSREG} Stata alternatives), this Julia algorithm allows researchers to obtain the best model specification among all possible 
covariate/feature combinations - in terms of user defined information criteria-, but up to 3165 times faster than Stata and 197 times faster than R. \vskip 6pt

% **************GENERATED FILE, DO NOT EDIT**************

\bibliographystyle{juliacon}
\bibliography{ref.bib}
\end{document}

% Inspired by the International Journal of Computer Applications template
